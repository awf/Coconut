\subsection{Translation from \lafsharp{} to \salafsharp}
\label{sec:fsmooth_to_dps}

\begin{figure*}[h]
\centering
% \infer{\expr{}=\exprind{0}\ \vmore{\expr}\tab \expr{}: \typemat \tab \typemat \neq \typedouble}
\begin{tabular}{l r c l}
& \stgtranss{\expr}{\stgvar} &=&
\exprdps{}
\\ \\
(D-App) & 
\stgtranss{\vapp{\exprind{0}}{\vvarind{1} ... \vvarind{k}}}{\stgvar}&=&
% \vallocterm{} (\cardwidth{\cardtrans{\vapp{\exprind{0}}{\vvarind{1} ... \vvarind{k}}}}) 
% (\vabs{\stgvarind{2}}{}\\
% & & & 
% \tab (\stgtranss{\exprind{0}}{\stgempty})\ \stgvarind{2}\ \stgvarpostfix{\vvarind{1}} ... \stgvarpostfix{\vvarind{k}}) \\
(\stgtranss{\exprind{0}}{\stgempty})\ \stgvar{}\ \stgvarpostfix{\vvarind{1}} ... \stgvarpostfix{\vvarind{k}}  \cardvarpostfix{\vvarind{1}} ... \cardvarpostfix{\vvarind{k}} \\
% \infer{\expr{}=\vabs{$\vmore{\text{x}}$}{\exprind{1}} \tab \expr{}: \typefun{\typet}{\typemat} \tab \typemat \neq \typedouble}
(D-Abs) &
\stgtranse{\vabs{\vvarind{1} ... \vvarind{k}}{\exprind{1}}}&=&
\vabs{\stgvarind{2} \stgvarpostfix{\vvarind{1}} ... \stgvarpostfix{\vvarind{k}}  \cardvarpostfix{\vvarind{1}} ... \cardvarpostfix{\vvarind{k}}}{ \stgtranss{\exprind{1}}{\stgvarind{2}} } \\
(D-VarScalar) & 
\stgtranse{\text{x}}&=&
\stgvarpostfix{\text{x}}\\
(D-VarVector) & 
\stgtranss{\text{x}}{\stgvar}&=&
\vcopys{\stgvar}{\stgvarpostfix{\text{x}}}\\
(D-Let) &
\stgtrans{\vlet{\text{x}}{\exprind{1}}{\exprind{2}}}&=&
\vlet{\cardvarpostfix{\text{x}}}{\cardtrans{\exprind{1}}}{}\\
& & & 
\vallocterm{} (\cardwidth{\cardvarpostfix{\text{x}}}) (\vabs{\stgvarind{2}}{}
\\
& & & 
\tabt \vlet{\stgvarpostfix{\text{x}}}{\stgtranss{\exprind{1}}{\stgvarind{2}}}{\stgtrans{\exprind{2}}})\\
(D-If) & 
\stgtrans{\vifthenelse{\exprind{1}}{\exprind{2}}{\exprind{3}}}&=&
\vifthenelse{\stgtranse{\exprind{1}}}{\stgtrans{\exprind{2}}}{\stgtrans{\exprind{3}}}\\
% (Tr-BuildEmpty) &
% \stgtranss{\vbuild{\exprind{0}}{\exprind{1}}}{\stgempty} &=&
% \vlet{\text{x}}{\stgtranss{\exprind{0}}{\stgempty}}{}\\
% & & &
% \withstg{\text{x}}{\stgvarind{2}}{\vbuilds{\stgvarind{2}}{\text{x}}{\stgtranss{\exprind{1}}{\stgempty}}}\\
% (D-Vector) & \stgtrans{\varr{\exprind{1}, ..., \exprind{\valcard}}} &=&
% \varrs{\stgvar}{\vabs{\stgvarind{1}}{\stgtranss{\exprind{1}}{\stgvarind{1}}}, ..., \vabs{\stgvarind{\valcard}}{\stgtranss{\exprind{\valcard}}{\stgvarind{\valcard}}}} \\
% \\
\\
& \stgtranstype{\typet} &=&
\typetdps
\\
\\
(DT-Fun) &
\stgtranstype{\typefunone{\typeind{1}, ..., \typeind{k}}{\typemat}} &=&
\typefunone{\typestg{}, \stgtranstype{\typeind{1}}, ..., \stgtranstype{\typeind{k}}, \cardtranstype{\typeind{1}}, ..., \cardtranstype{\typeind{k}}}{\stgtranstype{\typemat}} \\
(DT-Mat) &
\stgtranstype{\typemat} &=&
\typemat \\
(DT-Bool) &
\stgtranstype{\typebool} &=&
\typebool \\
(DT-Card) &
\stgtranstype{\typecard} &=&
\typecard \\
\end{tabular}
\caption{Translation from \lafsharp{} to \salafsharp{}}
\label{fig:salaf_trans}
\end{figure*}

% \begin{figure}
% \centering
% \begin{subfigure}[b]{.85\columnwidth}
% \lett{} add = \vabs{a b}{}
% \\
% \tabt
% \vbuild{(\vlength{a})}{(\vabs{i}{\vget{a}{i} + \vget{b}{i}})} \code{in}
% \\
% \vlet{vec1}{[1, 2, 3]}{}
% \\
% \vlet{vec2}{[4, 5, 6]}{}
% \\
% add vec1 vec2
% \caption{Initial Program in \lafsharp}
% \end{subfigure}
% \begin{subfigure}[b]{.85\columnwidth}
% \vlet{\cardvarpostfix{add}}{\vabscard{\cardvarpostfix{a} \cardvarpostfix{b}}{\cardvector{\cardempty}{\cardvectorsize{\cardvarpostfix{a}}}}}{}
% \\
% \vallocterm{} (\cardwidth{\cardvarpostfix{add}}) (\vabs{\stgvarind{1}}{}
% \\
% \tabt \lett{} \stgvarpostfix{add} = \vabs{\stgvarind{2} \stgvarpostfix{a} \stgvarpostfix{b}}{}
% \\
% \tabt \tabt \vbuilds{\stgvarind{2}}{(\vlength{\stgvarpostfix{a}})}{\vabs{\stgvarind{3} \stgvarpostfix{i}}{\vget{\stgvarpostfix{a}}{\stgvarpostfix{i}} + \vget{\stgvarpostfix{b}}{\stgvarpostfix{i}}}}
% \\
% \tabt \vlet{\cardvarpostfix{vec1}}{\cardvector{\cardempty}{3}}{}
% \\
% \tabt \vallocterm{} (\cardwidth{\cardvarpostfix{vec1}}) (\vabs{\stgvarind{4}}{}
% \\
% \tabt \tabt \vlet{\stgvarpostfix{vec1}}{\varrs{\stgvarind{4}}{\vabs{\_}{1}, \vabs{\_}{2}, \vabs{\_}{3}}}{}
% \\
% \tabt \tabt \vlet{\cardvarpostfix{vec2}}{\cardvector{\cardempty}{3}}{}
% \\
% \tabt \tabt \vallocterm{} (\cardwidth{\cardvarpostfix{vec2}}) (\vabs{\stgvarind{5}}{}
% \\
% \tabt \tabt \tabt \vlet{\stgvarpostfix{vec2}}{\varrs{\stgvarind{5}}{\vabs{\_}{4}, \vabs{\_}{5}, \vabs{\_}{6}}}{}
% \\
% \tabt \tabt \tabt
% \stgvarpostfix{add} \stgvarind{0} \stgvarpostfix{vec1} \stgvarpostfix{vec2}
% \\
% )\tabtm)\tabtm)
% \caption{Transformed Program in \salafsharp}
% \end{subfigure}
% \begin{subfigure}[b]{.85\columnwidth}
% \lett{} \stgvarpostfix{add} = \vabs{\stgvarind{2} \stgvarpostfix{a} \stgvarpostfix{b}}{}
% \\
% \tabt \vbuilds{\stgvarind{2}}{(\vlength{\stgvarpostfix{a}})}{\vabs{\stgvarind{3} \stgvarpostfix{i}}{\vget{\stgvarpostfix{a}}{\stgvarpostfix{i}} + \vget{\stgvarpostfix{b}}{\stgvarpostfix{i}}}}
% \\
% \vallocterm{} 6 (\vabs{\stgvarind{6}}{}
% \\
% \tabt \vlet{\stgvarpostfix{vec1}}{\varrsat{\stgvarind{6}}{1, 2, 3}}{}
% \\
% \tabt \vlet{\stgvarpostfix{vec2}}{\varrsat{(\stgoffset{} \stgvarind{6} 3)}{4, 5, 6}}{}
% \\
% \tabt
% \stgvarpostfix{add} \stgvarind{0} \stgvarpostfix{vec1} \stgvarpostfix{vec2}
% \\
% )
% \caption{More Optimized Program in \salafsharp}
% \end{subfigure}
% \caption{Example of Translation from \lafsharp{} to \salafsharp{}}
% \label{fig:salaf_trans_ex}
% \end{figure}

We now turn our attention to the translation from \lafsharp{} to \salafsharp{}.  Before translating \lafsharp{} expressions to their DPS form, the expressions should be transformed into a normal form similar to Administrative-Normal Form~\cite{flanagan1993essence} (ANF). In this representation, 
each subexpression of an application is either a constant value or a variable.
This greatly simplifies the translation rules, specially the (D-App) rule.\footnote{
In a true ANF representation, \textit{every} subexpression is a constant value or a variable,
whereas in our case, we only care about the subexpressions of an application.
Hence, our reprsentation is \textit{almost} ANF.
}
The representation of our working example in ANF is as follows:

\begin{figure}[H]
\hfill\begin{minipage}{.75\textwidth}\raggedright
f = \vabs{vec1 vec2}{}
\\
\tabt \vlet{tmp}{vectorAdd vec1 vec2}{}
\\
\tabt vectorNorm tmp
\end{minipage}\hfill
\end{figure}

Figure~\ref{fig:salaf_trans} shows the translation from \lafsharp{} to \salafsharp{}, where $\stgtranss{\expr}{\stgvar}$ is the translation of a \lafsharp{} expression $\expr$ into a \salafsharp{} expression that stores $\expr$'s value in memory $\stgvar$. Rule (D-Let) is a good place to start.  It uses $\vallocterm$ to allocate enough space for the value of $\exprind{1}$, the right hand side of the let --- but how much space is that?  We use an auxiliary translation $\cardtrans{\exprind{1}}$ to translate $\exprind{1}$ to an expression that computes $\exprind{1}$'s \emph{shape} rather than its \emph{value}. The shape of an array expression specifies the cardinality of each dimension. We will discuss why we need shape (what goes wrong with just using bytes) and the shape translation in Section~\ref{sec:card}.  This shape is bound to $\cardvarpostfix{x}$, and used in the argument to $\vallocterm$.  The freshly-allocated storage $\stgvarind{2}$ is used as 
the destination for translating the right hand side $\exprind{1}$, while the original destination $\stgvar$ is used as the destination for the body $\exprind{2}$. 

In general, every variable $x$ in \lafsharp{} becomes a \emph{pair} of variables $\stgvarpostfix{x}$ (for $x$'s value) and $\cardvarpostfix{x}$ (for $x$'s shape) in \salafsharp{}.  You can see this same phenomenon in
rules (D-App) and (D-Abs), which deal with lambdas and application: we turn each lambda-bound argument $x$ into \emph{two} arguments $\stgvarpostfix{x}$ and $\cardvarpostfix{x}$.

Finally, in rule (D-App) the destination memory $\stgvar$ for the context is passed on to the function being called, as its additional first argument; and in (D-Abs) each lambda gets an additional first argument, which is used as the destination when translating the body of the lambda.  Figure~\ref{fig:salaf_trans} also gives a translation of an \lafsharp{} type $\typet$ to the corresponding \salafsharp{} type $\typetdps$.


% As all expression are in ANF form, there is no need to apply the DPS transformation to the arguments of the application. 
% Furthermore, the applied function does not require any memory (i.e. \stgempty{} should be passed as the output storage). This is because there is no heap memory needed for expressions of function type.

% To prove this, we require presenting the following theorem.
% \begin{theorems}
% \label{theor2}
% In the application expression \vapp{\expr}{\vmore{\expr}} of \lafsharp{}, the subexpression \expr{} is either an abstraction (\vabs{\text{x}}{\expr}) or a variable access (\text{x}).
% \end{theorems}
% \text{Proof.} Immediate from the fact that the expressions are in ANF representation.

% Figure~\ref{fig:salaf_trans} shows the translation from \lafsharp{} to \salafsharp{}. There are two important points regarding the first rule for application.
% First, this rule uses Theorem \ref{no_heap_scalar_thrm} and Theorem \ref{theor2} in order to use an empty memory (\stgempty) for the function that is applied. Second, this rule assumes some sort of normal form (i.e. ANF) in which the arguments are let bound. This simplifies the specification of transformation rule. The rule for the original form of application can be derived by combining this rule and the rule for let binding.

% DPS version of $\lambda$ expressions involves adding two sets of arguments. First, a single argument representing the storage which stores the output result. Second, a list of arguments representing the shape information associated with the input arguments. The body of the $\lambda$ expression is converted to its DPS form using the output storage specified for this expression.

For variables there are two cases. In rule (D-VarScalar) a scalar variable is translated to itself, while in rule (D-VarVector) we must copy the array into the designated result storage
using the \cod{copy} function. The \cod{copy} function copies the array elements as well as the header information into the given storage. % First, a variable is a non-array type variable. In this case, the needed memory can be stack allocated and there is no heap memory needed for it. Second case is when the variable is of array type. As we would like to avoid aliasing the variables allocated in a particular scope, we copy that value into the given output storage.

%All allocation/deallocation constructs are introduced in the transformation rule for let binding. This fact clearly shows how translating \lafsharp{} expressions into their ANF representation simplifies the DPS translation rules. After translating the shape of the binding expression, we allocate the number of bytes specifies by that shape expression. This allocated memory is used for the DPS translation of the binding expression. The DPS version of the body of let binding uses the output storage provided for the whole let expression.

\subsection{Shape translation}
\label{sec:card}\label{sec:shapetrans}

\begin{figure*}
\centering
% \def\comment{\hfill -- }
% \begin{tabular}{r c l}
% \carddom{} & ::= & \carddommat{} \comment Matrix Cardinality \\
% & | & \typefun{\carddom}{\carddommat} \comment Function Cardinality \\
% & | & \cardempty \comment No Cardinality\\
% & | & \typecard \comment Cardinality\\
% \\
% \carddommat{} & ::= & \carddouble{} \comment Scalar Width\\
% & | & \cardvector{\carddommat{}}{\typecard} \comment Vector, Matrix, ... Cardinality\\
% \end{tabular}
% % \infer{\expr{}=\exprind{0}\ \vmore{\expr}\tab \expr{}: \typemat \tab \typemat \neq \typedouble}
% \vspace{1cm}

\begin{tabular}{l r c l}
% & \cardtrans{\lafsharp{} Expression} &=&
% \shapefsmooth{} Expression
& \cardtrans{\expr} &=&
\exprshape{}
\\ \\
(S-App) & 
\cardtrans{\vapp{\exprind{0}}{\exprind{1} ... \exprind{k}}
}&=&
\vapp{\cardtrans{\exprind{0}}}{\cardtrans{\exprind{1}} ... \cardtrans{\exprind{k}}}\\
(S-Abs) & 
\cardtrans{\vabs{$x_1$: $T_1$, ..., $x_k$: $T_k$}{\expr}
}&=&
\vabscard{\cardvarpostfix{$x_1$}: \cardtranstype{$T_1$}, ..., \cardvarpostfix{$x_k$}: \cardtranstype{$T_k$}}{\cardtrans{\expr}}\\
(S-Var) & 
\cardtrans{\text{x}}&=&
\cardvarpostfix{\text{x}}\\
(S-Let) & 
\cardtrans{\vlet{\text{x}}{\exprind{1}}{\exprind{2}}}&=&
\vlet{\cardvarpostfix{\text{x}}}{\cardtrans{\exprind{1}}}{\cardtrans{\exprind{2}}}\\
(S-If) & 
\cardtrans{\vifthenelse{\exprind{1}}{\exprind{2}}{\exprind{3}}}&=&
$\begin{cases}
    \cardtrans{\exprind{2}}       & \qquad \qquad \cardtrans{\exprind{2}} \shpeq{} \cardtrans{\exprind{3}}\\
    \text{Compilation Error!}  & \qquad \qquad \cardtrans{\exprind{2}} \shpneq{} \cardtrans{\exprind{3}}\\
  \end{cases}
$
\\
% (C-Vector) & \cardtrans{\varr{\exprind{1}, ..., \exprind{\valcard}}} &=&
% \cardvector{\cardtrans{\exprind{1}}}{\valcard} \\
% (S-Vector) &  \cardtrans{\varr{\exprind{1}, ..., \exprind{\valcard}}} &=&
% $\begin{cases}
%     \valcard{}       & \quad \forall k.\ \exprind{k}: \typenum{}\\
%       & \quad \forall k.\ \exprind{k}: \typearray{\typemat{}}
%     \\
%     \cardvector{\cardtrans{\exprind{1}}}{\valcard} & \quad \text{and}
%     \\
%     & \quad \forall k,k'.\ \cardtrans{\exprind{k}}\shpeq{}\cardtrans{\exprind{k'}}\\
%     \text{Compilation Error!} & \quad \text{otherwise}
%   \end{cases}
% $
% (S-Vector) &  \cardtrans{\varr{\exprind{1}, ..., \exprind{\valcard}}} &=&
% $\begin{cases}
%     \cardvector{\valcard}{\cardtrans{\exprind{1}}} & \quad \forall k,k'.\ \cardtrans{\exprind{k}}\shpeq{}\cardtrans{\exprind{k'}}
%     \\
%     \text{Compilation Error!} & \qquad \qquad \text{otherwise}
%   \end{cases}
% $
%  \\
% (C-Matrix) & \exprind{k}: \typearray{\typemat{}} $\vdash$ \cardtrans{\varr{\exprind{1}, ..., \exprind{\valcard}}} &=&
% \cardvector{\cardtrans{\exprind{1}}}{\valcard} \\
% (C-Build) &
% \cardtrans{\vbuild{\exprind{0}}{\exprind{1}}} &=&
% \cardvector{(\vapp{\cardtrans{\exprind{1}}}{\cardempty})}{\cardtrans{\exprind{0}}} \\
% (C-Get) &
% \cardtrans{\vget{\exprind{0}}{\exprind{1}}} &=&
% \cardvectorelem{\cardtrans{\exprind{0}}} \\
% (C-Length) &
% \cardtrans{\vlength{\exprind{0}}} &=&
% \cardvectorsize{\cardtrans{\exprind{0}}} \\
% (C-Fold) &
% \cardtrans{\vfold{\exprind{2}}{\exprind{3}}{\exprind{1}}} &=&
% \cardtrans{\exprind{2}} \\
\\
(S-ExpNum) &
\expr{}: \typenum{} $\vdash$
\cardtrans{\expr{}} &=&
\cardempty \\
(S-ExpBool) &
\expr{}: \typebool{} $\vdash$
\cardtrans{\expr{}} &=&
\cardempty \\
(S-ValCard) &
\cardtrans{\text{N}} &=&
\text{N} \\
(S-AddCard) &
\cardtrans{\exprind{0} \vcaddcard{} \exprind{1}} &=&
\cardtrans{\exprind{0}} \vcaddcard{} \cardtrans{\exprind{1}} \\
(S-MulCard) &
\cardtrans{\exprind{0} \vcmulcard{} \exprind{1}} &=&
\cardtrans{\exprind{0}} \vcmulcard{} \cardtrans{\exprind{1}} \\
(S-Build) &
\cardtrans{\vbuild{\exprind{0}}{\exprind{1}}}
 &=&
\cardvector{\cardtrans{\exprind{0}}}{(\vapp{\cardtrans{\exprind{1}}}{\cardempty})} \\
(S-Get) &
\cardtrans{\vget{\exprind{0}}{\exprind{1}}}
&=&
\cardvectorelem{\cardtrans{\exprind{0}}} \\
(S-Length) &
\cardtrans{\vlength{\exprind{0}}}
&=&
\cardvectorsize{\cardtrans{\exprind{0}}} \\
(S-Reduce) &
\cardtrans{
\viteratek{} \exprind{1} \exprind{2} \exprind{3}
} &=&
$\begin{cases}
    \cardtrans{\exprind{2}}       & \quad \forall n. \cardtrans{\exprind{1} \exprind{2} n} \shpeq{} \cardtrans{\exprind{2}}\\
    \text{Compilation Error!}  & \qquad \qquad \text{otherwise}\\
  \end{cases}
$
\\ \\
& \cardtranstype{\typet{}} &=& \typetshape{}
\\ \\

(ST-Fun) &
\cardtranstype{\typefunone{\typeind{1}, \typeind{2}, ..., \typeind{k}}{\typemat}} &=&
\typefunone{\cardtranstype{\typeind{1}}, \cardtranstype{\typeind{2}}, ..., \cardtranstype{\typeind{k}}}{\cardtranstype{\typemat}} \\
(ST-Num) &
\cardtranstype{\typenum} &=&
\typecard \\
(ST-Bool) &
\cardtranstype{\typebool} &=&
\typecard \\
(ST-Card) &
\cardtranstype{\typecard} &=&
\typecard \\
% (CT-Vector) &
% \cardtranstype{\typearray{\typemat}} &=&
% \cardvector{\cardtranstype{\typemat}}{\typecard} \\
% (ST-Vector) &
% \cardtranstype{\typearray{\typenum}} &=&
% {\typecard} \\
(ST-Vector) &
\cardtranstype{\typearray{\typemat}} &=&
\cardvector{\typecard}{\cardtranstype{\typemat}} \\
% \\
% (W-NoCard) &
% \cardwidth{\cardempty} &=&
% 0
% \\
% (W-Card) &
% \cardwidth{\valcard} &=&
% \valcard
% \\
% (W-Scalar) &
% \cardwidth{\carddouble} &=&
% 4
% \\
% (W-Vector) &
% \cardwidth{\cardvector{c}{\valcard}} &=&
% \cardwidth{c} $\times$ \valcard
% \\
\end{tabular}
\caption{Shape Translation of \lafsharp{}}
\label{fig:laf_card}
\end{figure*}

As we have seen, rule (D-Let) relies on the \emph{shape translation} of the right
hand side.  This translation is given in Figure~\ref{fig:laf_card}.
If $\expr$ has type $\typet$, then $\cardtrans{\expr}$ is an expression 
of type $\cardtranstype{\typet}$ that gives the shape of $\expr$.
This expression can always be evaluated without allocation.

A \emph{shape} is an expression of type $\carddom{}$ (Figure~\ref{fig:salaf_core_syntax}),
whose values are given by $\shapevar$ in that Figure.  There are three cases to consider

First, a scalar value has shape $\cardempty$ (rules (S-ExpNum), (S-ExpBool)).

Second, when $\expr$ is an array, $\cardtrans{\expr}$ gives the shape of the array as
a nested tuple, such as $(3,(4,\cardempty))$ for a 3-vector of 4-vectors.
So the ``shape'' of an array specifies the cardinality of each dimension.

Finally, when $\expr$ is a function, $\cardtrans{\expr}$ is a function that takes the shapes of its arguments and returns the shape of its result. 
You can see this directly in rule (S-App): to compute the shape of (the result of) a call, apply the shape-translation of the function to the shapes of the arguments.
This is possible because \lafsharp{} programs do not allow the programmer
to write a function whose result size depends on the contents of its input array.

What is the shape-translation of a function $\text{f}$?  Remembering that every in-scope variable $\text{f}$ has become a pair of variables one for the value and one for the shape, we can simply use the latter, $\cardvarpostfix{\text{f}}$, as we see in rule (S-Var).

For arrays, could the shape be simply the number of bytes required for the array,
rather than a nested tuple?  No.
Consider the following function, which returns the first row of its argument matrix:

firstRow = \vabs{m: \typearray{\typearray{\typedouble{}}}}{\vget{m}{0}}

The shape translation of firstRow, namely \cardvarpostfix{firstRow}, is given the shape of m, and must produce the
shape of m's first row.  It cannot do that given only the number of bytes in m; it must know how many rows
and columns it has.  But by defining shapes as a nested tuple, it becomes easy: see rule (S-Get).

The shape of the result of the iteration construct (\viteratek) requires the shape of 
the state expression to remain the same across iterations. Otherwise the compiler
produce an error, as it is shown in rule (S-Reduce).  

The other rules are straightforward.  \emph{The key point is this: by translating every in-scope variable, including functions, into a pair of variables, we can give a \emph{compositional} account of shape translation, even in a higher order language.}






\subsection{An example}

Using this translation, the running example at the beginning of Section~\ref{sec:fsmooth_to_dps}
is translated as follows:

\begin{figure}[H]
\hfill\begin{minipage}{.75\textwidth}\raggedright
% \vlet{\cardvarpostfix{add}}{\vabscard{\cardvarpostfix{a} \cardvarpostfix{b}}{\cardvector{\cardempty}{\cardvectorsize{\cardvarpostfix{a}}}}}{}
% \\
% \vallocterm{} (\cardwidth{\cardvarpostfix{add}}) (\vabs{\stgvarind{1}}{}
% \\
% \tabt \lett{} \stgvarpostfix{add} = \vabs{\stgvarind{2} \stgvarpostfix{a} \stgvarpostfix{b}}{}
% \\
% \tabt \tabt \vbuilds{\stgvarind{2}}{(\vlength{\stgvarpostfix{a}})}{\vabs{\stgvarind{3} \stgvarpostfix{i}}{\vget{\stgvarpostfix{a}}{\stgvarpostfix{i}} + \vget{\stgvarpostfix{b}}{\stgvarpostfix{i}}}}
% \\
f = \vabs{\stgvarind{0} \stgvarpostfix{vec1} \stgvarpostfix{vec2} \cardvarpostfix{vec1} \cardvarpostfix{vec2}}{}
\\
\tabt \vlet{\cardvarpostfix{tmp}}{\cardvarpostfix{vectorAdd} \cardvarpostfix{vec1} \cardvarpostfix{vec2}}{}
\\
\tabt \vallocterm{} (\cardwidth{\cardvarpostfix{tmp}}) (\vabs{\stgvarind{1}}{}
\\
\tabt \tabt \lett{} \stgvarpostfix{tmp} =
\\
\tabt \tabt \tabt 
\stgvarpostfix{vectorAdd} \stgvarind{1} \stgvarpostfix{vec1} \stgvarpostfix{vec2}
\\
\tabt \tabt \tabt \tabt \tabt \cardvarpostfix{vec1} \cardvarpostfix{vec2} \code{in}
\\
\tabt \tabt 
\stgvarpostfix{vectorNorm} \stgvarind{0} \stgvarpostfix{tmp} \cardvarpostfix{tmp}
\\
\tabt )
\end{minipage}\hfill
\end{figure}

The shape translations of some \lafsharp{} functions from Figure~\ref{fig:smooth_lib} are as follows:

\begin{figure}[H]
\hfill\begin{minipage}{.75\textwidth}\raggedright
\lett{} \cardvarpostfix{vectorRange} = \vabscard{\cardvarpostfix{n}}{}
% \cardvarpostfix{\code{build}} \cardvarpostfix{n} (\vabscard{\cardvarpostfix{i}}{\cardempty})
\cardvector{\cardvarpostfix{n}}{\vapp{(\vabscard{\cardvarpostfix{i}}{\cardempty})}{ \cardempty}}
\\
\lett{} \cardvarpostfix{vectorMap2} = \vabscard{\cardvarpostfix{v1} \cardvarpostfix{v2} \cardvarpostfix{f}}{}
\\
\tabt 
% \cardvarpostfix{\code{build}} (\cardvarpostfix{\code{length}}  \cardvarpostfix{v1}) (\vabscard{\cardvarpostfix{i}}{\cardempty})
\cardvector{\cardvectorsize{\cardvarpostfix{v1}}}{\vapp{(\vabscard{\cardvarpostfix{i}}{\cardempty})}{ \cardempty}}
\\
\lett{} \cardvarpostfix{vectorAdd} = \vabscard{\cardvarpostfix{v1} \cardvarpostfix{v2}}{}
\\
\tabt
\cardvarpostfix{vectorMap2} \cardvarpostfix{v1} \cardvarpostfix{v2}
(\vabscard{\cardvarpostfix{a} \cardvarpostfix{b}}{\cardempty})
\\
\lett{} \cardvarpostfix{vectorNorm} = \vabscard{\cardvarpostfix{v}}{\cardempty}
\end{minipage}\hfill
\end{figure}

\subsection{Simplification}
\label{sec_simplification}
\begin{figure}
\centering
\begin{tabular}{l c l r}
% % (EQ-BuildsGets)&
% \vgets{\stgvarind{2}}{(\vbuilds{\stgvarind{1}}{\exprind{0}}{\exprind{1}})}{\exprind{2}} 
% &\evalsto&
% \exprind{1}\ \stgvarind{2}\   \exprind{2} \\
% % (EQ-BuildsLength)&
% \vlength{(\vbuilds{\stgvar}{\exprind{0}}{\exprind{1}})} 
% &\evalsto&
% \exprind{0} \\
% % (EQ-VectorsGets)&
% \vgets{\stgvarind{2}}{\varrs{\stgvarind{1}}{\exprind{1}, ..., \exprind{\valcard}}}{\text{i}} 
% &\evalsto&
% \exprind{\text{i}} \ \stgvarind{2} \\
% % (EQ-VectorsLength)&
% \vlength{\varrs{\stgvar}{\exprind{1}, ..., \exprind{\valcard}}}
% &\evalsto&
% \valcard\\
% (EQ-AllocEmpty)&
\withstg{\cardempty}{\stgvar}{\exprdpsind{1}}
&\evalsto&
\exprdpsind{1}[\stgvar{} $\mapsto$ \stgempty]
& Empty Allocation\\
% (EQ-AllocMerge)&
\vallocterm{} \exprdpsind{1} (\vabs{\stgvarind{1}}{}
% \withstg{\valcardind{1}}{\stgvarind{1}}{\withstg{\valcardind{2}}{\stgvarind{2}}{\exprind{1}}}
&\evalsto&
\vallocterm{} (\exprdpsind{1} \vcaddcard{} \exprdpsind{2}) (\vabs{\stgvarind{1}}{}
\\
% & 
\tab
\vallocterm{} \exprdpsind{2} (\vabs{\stgvarind{2}}{}
& & \tab 
\vlet{\stgvarind{2}}{\stgoffset{} \stgvarind{1} \exprdpsind{1}}{}
& Allocation Merging
\\
\tab \tab \exprdpsind{3} ))
& & \tab
\exprdpsind{3} )
\\
% \vgets{\stgvar{}}{\exprdpsind{0}}{\exprdpsind{1}}
% &\evalsto&
% \vgets{\stgempty}{\exprdpsind{0}}{\exprdpsind{1}}
% \\
\withstg{\exprdpsind{1}}{\stgvar}{\exprdpsind{2}}
&\evalsto&
\exprdpsind{2}
\hfill \text{if $\stgvar{} \notin FV(\exprdpsind{2})$}
& Dead Allocation
\\
$\lambda x. $
\withstg{\exprdpsind{1}}{\stgvar}{\exprdpsind{2}}
&\evalsto&
\withstg{\exprdpsind{1}}{\stgvar}{$\lambda x. $ \exprdpsind{2}}
\hfill \text{if $x \notin FV(\exprdpsind{1})$}
& Allocation Hoisting
\\
\cardwidth{\cardempty} 
& \evalsto &
\cardempty 
\\
\cardwidth{\cardvector{\cardempty}{\cardempty}} 
& \evalsto &
\cardempty 
\\
\cardwidth{\cardvector{\valcard}{\cardempty}} 
& \evalsto & 
\vectorheader{\vectorelembytes{} \vcmulcard{} \valcard}
\\
\cardwidth{\cardvector{\valcard}{\exprshape}} 
& \evalsto & 
\vectorheader{(\cardwidth{\exprshape}) \vcmulcard{} \valcard}
\end{tabular}
\caption{Simplification rules of \salafsharp{}}
\label{fig:salaf_eq}
\end{figure}

As is apparent from the examples in the previous section,
code generated by the translation has many optimisation opportunities.
This optimisation, or simplification, is applied in three stages: 1) \lafsharp{} expressions, 2) translated \shapefsmooth{} expressions, and 3) translated \salafsharp{} expressions. In the first stage, \lafsharp{} expressions are simplified to exploit fusion opportunities that remove intermediate arrays entirely. Furthermore, other compiler transformations such as constant folding, dead-code elimination, and common-subexpression elimination are also applied at this stage.

In the second stage, the \shapefsmooth{} expressions are simplified. The simplification process for these expressions mainly involves partial evaluation. By inlining all shape functions, and performing $\beta$-reduction and constant folding, shapes can often be computed at compile time, or at least can be greatly simplified. For example, the shape translations presented in Section~\ref{sec:shapetrans} after performing simplification are as follows:


\begin{figure}[H]
\hfill\begin{minipage}{.75\textwidth}\raggedright
\lett{} \cardvarpostfix{vectorRange} = \vabscard{\cardvarpostfix{n}}{\cardvector{\cardvarpostfix{n}}{\cardempty}}
\\
\lett{} \cardvarpostfix{vectorMap2} = \vabscard{\cardvarpostfix{v1} \cardvarpostfix{v2} \cardvarpostfix{f}}{
\cardvarpostfix{v1}
}
\\
\lett{} \cardvarpostfix{vectorAdd} = \vabscard{\cardvarpostfix{v1} \cardvarpostfix{v2}}{
\cardvarpostfix{v1}
}
\\
\lett{} \cardvarpostfix{vectorNorm} = \vabscard{\cardvarpostfix{v}}{\cardempty}
\end{minipage}\hfill
\end{figure}

The final stage involves both partially evaluating the shape expressions in \salafsharp{} and simplifying the storage accesses in the \salafsharp{} expressions. Figure~\ref{fig:salaf_eq} demonstrates simplification rules for storage accesses. The first two rules remove empty allocations 
and merge consecutive allocations, respectively. The third rule removes a dead allocation, 
i.e. an allocation for which its storage is never used. The fourth rule hoists an allocation outside 
an abstraction whenever possible. The benefit of this rule is amplified more in the 
case that the storage is allocated inside a loop (\vcbuild{} or \viteratek{}). Note that none of these
transformation rules are available in \lafsharp{}, due to the lack of explicit storage facilities.

After applying the presented simplification process, out working example is translated to the following program:
\begin{figure}[H]
\hfill\begin{minipage}{.75\textwidth}\raggedright
f = \vabs{\stgvarind{0} \stgvarpostfix{vec1} \stgvarpostfix{vec2} \cardvarpostfix{vec1} \cardvarpostfix{vec2}}{}
\\
\tabt \vallocterm{} (\code{bytes} \cardvarpostfix{vec1}) (\vabs{\stgvarind{1}}{}
\\
\tabt \tabt \lett{} \stgvarpostfix{tmp} =
\\
\tabt \tabt \tabt 
\stgvarpostfix{vectorAdd} \stgvarind{1} \stgvarpostfix{vec1} \stgvarpostfix{vec2} \\
\tabt \tabt \tabt \tabt \tabt \cardvarpostfix{vec1} \cardvarpostfix{vec2} \code{in}
\\
\tabt \tabt 
\stgvarpostfix{vectorNorm} \stgvarind{0} \stgvarpostfix{tmp} \cardvarpostfix{vec1}
\\
\tabt )
\end{minipage}\hfill
\end{figure}
\noindent
In this program, there is no shape computation at runtime. 


% Next, we show that the presented DPS transformation preserves the semantic of the original program.

% \subsection{Correctness}
% In order to show the correctness of the translation process, we present the operational semantics of \lafsharp{} and \salafsharp{} in Figures~\ref{fig:laf_red} and~\ref{fig:salaf_red}. Based on the operation semantics of the two languages and the translation scheme from \lafsharp{} to \salafsharp{} the following theorem can be stated:
% \begin{theorems}
% \label{laissala}
% If \expr{} is an expression with an array type (vector or matrix) in \lafsharp{}, then for all inputs and a large enough memory location \stgvar{}, the expression \stgtrans{\expr} will always produce a value which is equivalent to the result of evaluation of \expr{}.
% \end{theorems}
% % \textit{Proof.} Structural induction on the operation semantic rules of the \lafsharp{} expression with array type. For each of these rules, one should match the corresponding translation rule from Figure~\ref{fig:salaf_trans} and check the evaluation rule of the result \salafsharp{} expression from Figure~\ref{fig:salaf_red}.


% 
\begin{figure*}
\centering
\begin{subfigure}[t]{\columnwidth}

\def\comment{\hfill -- }
\begin{flushleft}
\textbf{Value Syntax:}\\
\end{flushleft}
\centering
\begin{tabular}{r c l}
\valvar & ::= & $\lambda \vmore{\text{x}}. \expr$ \comment Abstraction\\
& | & \text{n} \comment Scalar Value\\
& | & \text{i} \comment Index Value\\
& | & \valcard \comment Cardinality Value\\
& | & \varr{\valvar, ..., \valvar} \comment Vector Value\\
\end{tabular}
\\
\begin{flushleft}
\textbf{Computation Rules:}\\
\end{flushleft}
\begin{tabular}{r c l}
% (E-IfTrue) &
\vifthenelse{true}{\exprind{2}}{\exprind{3}} 
&\evalsto&
\exprind{2} \\
% (E-IfFalse) &
\vifthenelse{false}{\exprind{2}}{\exprind{3}} 
&\evalsto&
\exprind{3} \\
% (E-AppBeta) &
$\lambda$ \vmore{\text{x}}. \exprind{1}\ \vmore{\text{v}} 
&\evalsto&
\exprind{1}[\vmore{\text{x}} $\mapsto$ \vmore{\text{v}}] \\
% (E-LetBeta)& 
\text{\lett{} x = \text{v} \inn{} \exprind{2}} 
&\evalsto&
\exprind{2}[\text{x} $\mapsto$ \text{v}]\\
% (E-BuildEval)&
\vbuild{\valcard}{\text{v}} 
&\evalsto&
\varr{\text{v}\ 1, ..., \text{v}\ \valcard} \\
% (E-GetEval)&
\vget{\varr{\valvarind{1}, ..., \valvarind{\valcard}}}{\text{i}} 
&\evalsto&
\valvarind{\text{i}} \\
% (E-LengthEval)&
\vlength{\varr{\valvarind{1}, ..., \valvarind{\valcard}}}
&\evalsto&
\valcard \\
% (E-FoldStep)&
$\forall \valcard > 0.$ \viterate{\valvarind{1}}{\valvarind{2}}{\valcard} 
&\evalsto& \viterate{\valvarind{1}}{(\valvarind{1} \valvarind{2} (\valcard{} - 1))}{(\valcard{} - 1)}
\\
% (E-FoldLast)&
\viterate{\valvarind{1}}{\valvarind{2}}{0} 
&\evalsto&
\valvarind{2}\\
\end{tabular}
% \subcaption{Computation Rules}

\end{subfigure}
\begin{subfigure}[t]{\columnwidth}
% \newcommand{\comment}[1]{\hfill -- #1}
\newcommand{\comment}[1]{}
\begin{flushleft}
\textbf{Evaluation Context:}\\
\end{flushleft}
\centering
\begin{tabular}{r c l}
\evalctx{} & ::= & \comment{Evaluation Context}\\
& | & [] \comment{ Evaluation Hole}\\
& | & \vifthenelse{\evalctx}{\exprind{2}}{\exprind{3}} \comment{If Condition} \\
& | & \vapp{\evalctx}{$\vmore{\expr}$} \comment{Application Function}\\
& | & \vapp{\valvarind{0}}{\valvarind{1} ... \valvarind{k-1} \evalctx{} \exprind{k+1} ... \exprind{N}} \comment{Application Arguments}\\
& | & \vlet{\text{x}}{\evalctx}{\exprind{1}} \comment{Let Binding}\\
& | & \varr{\valvarind{1}, ..., \valvarind{k-1}, \evalctx{}, \exprind{k+1}, ..., \exprind{N}} \comment{Vector Construct}\\
% & | & \vbuild{\evalctx}{\exprind{1}} \comment{Build Size}\\
% & | & \vbuild{\valvarind{0}}{\evalctx} \comment{Build Function}\\
% & | & \vget{\evalctx}{\exprind{1}} \comment{Get Vector}\\
% & | & \vget{\valvarind{0}}{\evalctx} \comment{Get Index}\\
% & | & \vlength{\evalctx} \comment{Length}\\
% & | & \vfold{\exprind{2}}{\exprind{3}}{\evalctx} \comment{Fold Range}\\
% & | & \vfold{\evalctx}{\exprind{3}}{\valvarind{1}} \comment{Fold Init}\\
% & | & \vfold{\valvarind{2}}{\evalctx}{\valvarind{1}} \comment{Fold Function}
\\ \\
& & %\hspace*{0.2cm}
\infer{\expr{} \evalsto \expr{}\vprime}{\evalctx[\expr{}] \evalsto \evalctx[\expr{}\vprime]
}
\end{tabular}

% \subcaption{Congruence Rules}
\end{subfigure}

\caption{Operational Semantics of \lafsharp{}}
\label{fig:laf_red}
\end{figure*}
% \begin{figure*}
\def\comment{\hfill -- }
\centering
\begin{subfigure}[b]{\columnwidth}
\begin{tabular}{r c l}
\expr & ::= & ...\\
& | & \stgctx{\expr}{\stgvar} \comment Storage Context\\
\\
$\Delta$ & ::= & \stgmapping{\stgvar}{\stgarea{\text{w}}} \comment Uninitialized storage\\
& | & \stgmapping{\stgvar}{\text{v}} \comment Filled storage\\
\end{tabular}
\end{subfigure}
\begin{subfigure}[b]{\columnwidth}
\begin{tabular}{r c l}
\text{w} & ::= & \text{w} $\times$ \text{w}\\
& | & \valcard\\
& | & \widthexpr{\expr}\\ \\
\valvar & ::= & \vabs{\vmore{\text{x}}}{\expr} \comment Storage-Aware Abstraction\\
% & | & \vabscard{\vmore{\text{x}}}{\expr} \comment Cardinality Abstraction \\
& | & \stgval \comment Memory Location\\
& | & \stgempty \comment Empty Memory Location\\
& | & \text{n} \comment Scalar Value\\
& | & \text{i} \comment Index Value\\
& | & \shapevar \comment Shape Value\\
& | & \varrsat{\stgvar}{...\varr{\text{n}, ..., \text{n}}...} \comment Vector Value\\
\end{tabular}
\end{subfigure}
\begin{subfigure}[b]{\textwidth}
\vspace{1cm}
\centering
\begin{tabular}{l r c l}
(E-IfTrue) &
\vifthenelse{true}{\exprind{2}}{\exprind{3}} 
&\evalsto&
\exprind{2} \\
(E-IfFalse) &
\vifthenelse{false}{\exprind{2}}{\exprind{3}} 
&\evalsto&
\exprind{3} \\
(E-AppBeta) &
$\Delta, \stgmapping{\stgvarind{1}}{\stgarea{\widthexpr{\exprind{1}}}} \vdash$
$(\lambda \stgvar{} \vmore{\text{x}}. \exprind{1})\ \stgvarind{1} \ \vmore{\text{v}}$
&\evalsto&
$\Delta, \stgmapping{\stgvarind{1}}{\stgarea{\widthexpr{\exprind{1}}}} \vdash$
$\exprind{1}[\stgvar{} \mapsto \stgvarind{1}, \vmore{\text{x}} \mapsto \vmore{\text{v}}]$ \\
(E-LetBeta)& 
\text{\lett{} x = \text{v} \inn{} \exprind{2}} 
&\evalsto&
\exprind{2}[\text{x} $\mapsto$ \text{v}]\\
(E-Alloc) &
$\Delta \vdash$ \withstg{\valvarind{0}}{\stgvar}{\exprind{1}}
&\evalsto&
$\Delta, \stgmapping{\stgvar}{\stgarea{\valvarind{0}}} \vdash$
\stgctx{\exprind{1}}{\stgvar} \\
(E-BuildsEval)&
$\Delta, \stgmapping{\stgvar}{\stgarea{\widthexpr{\text{v} 1} \times \valcard}} \vdash$ \vbuilds{\stgvar}{\valcard}{\text{v}} 
&\evalsto& \\
& \multicolumn{3}{r}{
$\Delta, \stgmapping{\stgvar}{\stgarea{\widthexpr{\text{v} 1} \times \valcard}} \vdash$
\varrs{\stgvar}{\vabs{\stgvarind{1}}{\text{v}\ \stgvarind{1}\ 1}, ..., \vabs{\stgvarind{\valcard}}{\text{v}\ \stgvarind{\valcard}\ \valcard}}
} \\
(E-VectorEval)&
\multicolumn{3}{l}{
$\Delta, \stgmapping{\stgvar}{\stgarea{\widthexpr{\valscalarind{1}} \times \valcard}} \vdash$ 
\varrs{\stgvar}{\vabs{\stgvarind{1}}{\valscalarind{1}}, ...,  \vabs{\stgvarind{\valcard}}{\valscalarind{\valcard}}}
}
\\
&
&\evalsto&
$\Delta, \stgmapping{\stgvar}{\varr{\valscalarind{1..\valcard}}} \vdash$
\varrsat{\stgvar}{\valscalarind{1..\valcard}} \\
(E-MatrixEval)&
\multicolumn{3}{l}{
$\Delta, \stgmapping{\stgvar}{\stgarea{\widthexpr{\valscalarind{1,1}} \times \valcard1 \times \valcard2}} \vdash$ 
\varrs{\stgvar}{\vabs{\stgvarind{1}}{\varrsat{\stgvarind{1}}{\valscalarind{1,1}, ...}}, ...,  \vabs{\stgvarind{\valcard1}}{\varrsat{\stgvarind{\valcard1}}{..., \valscalarind{\valcard1, \valcard2}}}}
}
\\
&
&\evalsto&
\\
\multicolumn{4}{r}{
$\Delta, \stgmapping{\stgvar}{\varr{\varr{\valscalarind{1,1..\valcard1, \valcard2}}}} \vdash$
\varrsat{\stgvar}{\varr{\valscalarind{1,1..\valcard1, \valcard2}}}
}\\
(E-Free)&
$\Delta, \stgmapping{\stgvarind{2}}{\stgarea{?}} \vdash$ 
\stgctx{\varrsat{\stgvarind{1}}{\valscalarind{1..\valcard}}}{\stgvarind{2}}
&\evalsto&
$\begin{cases}
    \Delta \vdash 
\varrsat{\stgvarind{1}}{\valscalarind{1..\valcard}}       & \stgvarind{1} \neq \stgvarind{2}\\
    \text{Runtime Error!}  & \stgvarind{1} = \stgvarind{2}\\
  \end{cases}
$ \\
(E-GetsEvalV)&
\multicolumn{3}{l}{
$\Delta, \stgmapping{\stgvarind{1}}{\stgarea{\widthexpr{\valscalarind{1,1}} \times \valcard2}} \vdash$ \vgets{\stgvarind{1}}{(\varrsat{\stgvarind{2}}{\varr{\valscalarind{1,1}, ..., \valscalarind{\valcard1, \valcard2}}})}{\text{i}}
}
\\
&
&\evalsto&
$\Delta, \stgmapping{\stgvarind{1}}{\valscalarind{i,1..i,\valcard2}} \vdash$ \varrsat{\stgvarind{1}}{\valscalarind{i,1..i,\valcard2}}\\
(E-GetsEvalS)&
\vgets{\stgempty}{(\varrsat{\stgvar}{\valscalarind{1}, ..., \valscalarind{\valcard}})}{\text{i}} 
&\evalsto&
\valscalarind{\text{i}} \\
(E-LengthEvalV)&
\vlength{(\varrsat{\stgvar}{\varr{\valscalarind{1,1}, ..., \valscalarind{\valcard1,\valcard2}}})}
&\evalsto&
\valcard1 \\
(E-LengthEvalS)&
\vlength{(\varrsat{\stgvar}{\valscalarind{1}, ..., \valscalarind{\valcard}})}
&\evalsto&
\valcard \\
% (E-FoldsStep)&
% \vfolds{\stgvarind{2}}{\valvarind{2}\vprime}{\valvarind{3}\vprime}{(\varrsat{\stgvarind{1}}{\valscalarind{1}, ..., \valscalarind{\valcard}})} 
% &\evalsto& \\
% & 
% \multicolumn{3}{r}{
% \vfolds{\stgvarind{2}}{\valvarind{3}\vprime\ \stgvarind{2} \valvarind{2}\vprime\ \valvarind{1}}{\valvarind{3}\vprime}{(\varrsat{(\stgoffset{} \stgvarind{1} 1)}{\valscalarind{2}, ..., \valscalarind{\valcard}})}
% }\\
% (E-FoldsLastV)&
% $\Delta, \stgmapping{\stgvar}{\stgarea{\widthexpr{\valvarind{2}}}} \vdash$ \vfolds{\stgvar}{\valvarind{2}}{\valvarind{3}}{\varr{}} 
% &\evalsto&
% $\Delta, \stgmapping{\stgvar}{\valvarind{2}} \vdash$ 
% \valvarind{2}\\
% (E-FoldsLastS)&
% $\valvarind{2}: \typedouble \vdash$
% \vfolds{\stgempty}{\valvarind{2}}{\valvarind{3}}{\varr{}} 
% &\evalsto&
% \valvarind{2}\\
(E-ReducesStep)&
$\valcard > 0 \vdash$ \viterates{\stgvar}{\valvarind{1}}{\valvarind{2}}{\valcard} 
&\evalsto& \viterates{\stgvar}{\valvarind{1}}{(\valvarind{1}\ \stgvar{} \valvarind{2}\ (\valcard{} - 1))}{(\valcard{} - 1)}
\\
(E-ReducesLastV)&
$\Delta, \stgmapping{\stgvar}{\stgarea{\widthexpr{\valvarind{2}}}} \vdash$ \viterates{\stgvar}{\valvarind{1}}{\valvarind{2}}{0} 
&\evalsto&
$\Delta, \stgmapping{\stgvar}{\valvarind{2}} \vdash$ 
\valvarind{2}\\
(E-ReducesLastS)&
$\valvarind{2}: \typedouble \vdash$
\viterates{\stgempty}{\valvarind{1}}{\valvarind{2}}{0} 
&\evalsto&
\valvarind{2}\\
\end{tabular}
\subcaption{Computation Rules}

\end{subfigure}
\begin{subfigure}[b]{\textwidth}
\centering
\begin{tabular}{r c l}
\evalctx{} & ::= & \comment Evaluation Context\\
& | & [] \comment Evaluation Hole\\
& | & \vifthenelse{\evalctx}{\exprind{2}}{\exprind{3}} \comment If Condition \\
& | & \vapp{\evalctx}{\stgvar{} $\vmore{\expr}$} \comment Application Function\\
& | & \vapp{\valvarind{0}}{\stgvar{} \valvarind{1} ... \valvarind{k-1} \evalctx{} \exprind{k+1} ... \exprind{N}} \comment Application Arguments\\
& | & \vlet{\text{x}}{\evalctx}{\exprind{1}} \comment Let Binding\\
& | & \withstg{\evalctx}{\stgvar}{\exprind{1}} \comment Alloc Length\\
% & | & \withstg{\valvarind{0}}{\stgvar}{\evalctx} \comment Alloc Body\\
& | & \varrs{\stgvar}{\vabs{\stgvarind{1}}{\valvarind{1}}, ...,
% \vabs{\stgvarind{k-1}}{\valvarind{k-1}},
\vabs{\stgvarind{k}}{\evalctx{}}, 
% \vabs{\stgvarind{k+1}}{\exprind{k+1}},
..., \vabs{\stgvarind{\valcard}}{\exprind{\valcard}}} \comment Vector Construct\\
% & | & \vbuilds{\stgvar}{\evalctx}{\exprind{1}} \comment Build Size\\
% & | & \vbuilds{\stgvar}{\valvarind{0}}{\evalctx} \comment Build Function\\
% & | & \vgets{\stgvar}{\evalctx}{\exprind{1}} \comment Get Vector\\
% & | & \vgets{\stgvar}{\valvarind{0}}{\evalctx} \comment Get Index\\
% & | & \vlength{\evalctx} \comment Length\\
% & | & \vfolds{\stgvar}{\exprind{2}}{\exprind{3}}{\evalctx} \comment Fold Range\\
% & | & \vfolds{\stgvar}{\evalctx}{\exprind{3}}{\valvarind{1}} \comment Fold Init\\
% & | & \vfolds{\stgvar}{\valvarind{2}}{\evalctx}{\valvarind{1}} \comment Fold Function \\
& | & \stgctx{\evalctx}{\stgvar} \comment Storage Context
\\
\\ \\
& & \hspace*{1.7cm}
\infer{\Delta \vdash \expr{} \evalsto \Delta\vprime{} \vdash \expr{}\vprime}{\Delta \vdash \evalctx[\expr{}] \evalsto \Delta\vprime{} \vdash \evalctx[\expr{}\vprime]
}
\end{tabular}
\\
\subcaption{Congruence Rules}
\end{subfigure}
\caption{Operational Semantics of \salafsharp{}}
\label{fig:salaf_red}
\end{figure*}

\subsection{Properties of shape translation}
\label{sec:shape-properties}

The target language of shape translation is a subset of \salafsharp{} called \shapefsmooth{}.
The syntax of the subset is given in Figure~\ref{fig:shape_syntax}. It includes nested pairs, of statically-known depth, to represent shapes, but it does not include vectors.
That provides an important property for \shapefsmooth{} as follows:

\begin{theorems}
\label{theorcardheap}
All expressions resulting from shape translation, do not require any heap memory allocation.
\end{theorems}
\textit{Proof.} All the non-shape expressions have either scalar or function type. As it is shown in Figure~\ref{fig:laf_card} all scalar type expressions are translated into zero cardinality (\cardempty), which can be stack-allocated. On the other hand, the function type expressions can also be stack allocated. This is because we avoid partial application. Hence, the captured environment in a closure does not escape its scope. Hence, the closure environment can be stack allocated. Finally, the last case consists of expressions which are the result of shape translation for vector expressions. As we know the number of dimensions of the original vector expressions, the translated expressions are tuples with a known-depth, which can be easily allocated on stack. 

Next, we show the properties of our translation algorithm. First, let us investigate the impact of shape translation on \lafsharp{} types. For array types, we need to represent the shape in terms of the shape of each element of the array, and the cardinality of the array. We encode this information as a tuple. For scalar type and cardinality type expressions, the shape is a cardinality expression. This is captured in the following theorem:

\begin{theorems}
\label{theorcardtype}
If the expression \expr{} in \lafsharp{} has the type \typet{}, then \cardtrans{\expr} has type \cardtranstype{\typet}.
\end{theorems}
\textit{Proof.} Can be proved by induction on the translation rules from \lafsharp{} to \shapefsmooth{}.

% For the \lafsharp{} expressions with scalar types (such as numeric and boolean) the shape translation is quite straightforward, as specified by the following theorem.

% \begin{theorems}
% \label{theorcardnumber}
% The expressions of numeric or boolean type in \lafsharp{} have a zero cardinality (\cardempty).
% \end{theorems}
% \textit{Proof.} Obvious from the definition.

\begin{figure}[t]
\centering
\def\comment{\hfill -- }
\begin{tabular}{r c l}
% Prog & ::= & $\overline{\text{\cod{let} \text{x} = \expr{}}}$ \inn{} \expr{}
% \\
\exprshape & ::= & \exprshape{} $\vmore{\exprshape}$ |
\vabscard{\vmore{\text{x}}}{\exprshape} | 
\text{x} |
\shapevar{} |
\text{c} |
\lett{} x = \exprshape{} \inn{} \exprshape{} \\
\shapevar & ::= & \cardempty{} | 
\valcard{} | 
\cardvector{\valcard}{\shapevar} \\
\text{c} & ::= & \vcvecshape{} |
\cardvectorsize{} |
\cardvectorelem{} | \vcaddcard{} | \vcmulcard{} \\
\typetshape{} & ::= & \typefun{\typetshape}{\carddom} | 
\carddom \\
\carddom{} & ::= & \typecard{} |
\typenestedcard{\typecard{}}{\carddom{}}\\
\end{tabular}
\caption{\shapefsmooth{} syntax, which is a subset of the syntax of \salafsharp{} presented in Figure~\ref{fig:salaf_core_syntax}.}
\label{fig:shape_syntax}
\end{figure}

In order to have a simpler shape translation algorithm as well as better guarantees about the expressions resulting from shape translation, two important restrictions are applied on \lafsharp{} programs. 
\begin{enumerate}
\item The accumulating function which is used in the \viteratek{} operator should preserve the shape of the initial value. Otherwise, converting the result shape into a closed-form polynomial expression requires solving a recurrence relation.
\item The shape of both branches of a conditional should be the same. 
\end{enumerate}
These two restrictions simplify the shape translation as is shown in Figure~\ref{fig:laf_card}.


\begin{theorems}
\label{theorshapelintime}
All expressions resulting from shape translation require linear computation time with respect to the size of terms in the original \lafsharp{} program.
\end{theorems}
\textit{Proof.} This can be proved in two steps. First, translating a  \lafsharp{} expression into its shape expression, leads to an expression with smaller size. This can be proved by induction on translation rules. Second, the run time of a shape expression is linear in terms of its size. An important case is the \viteratek{} construct, which by applying the mentioned restrictions, we ensured their shape can be computed without any need for recursion. 

Finally, we believe that our translation is correct based on our successful implementation. 
However, we leave a formal semantics definition and the proof of correctness of the transformation as future work. 

