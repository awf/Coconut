\section{Destination-Passing Style}
\label{sec:salafsharp}

\begin{figure}[!t]
\centering
\def\comment{\hfill -- }
\begin{tabular}{r c l}
% Prog & ::= & $\overline{\text{\cod{let} \text{x} = \expr{}}}$ \inn{} \expr{}
% \\
\exprdps & ::= & \exprdps{} $\vmore{\exprdps}$ \comment Application\\
& | & \vabs{\vmore{\text{x}}}{\exprdps} \comment Abstraction\\
& | & \stgempty \comment Empty Memory Location\\
& | & \text{n} \comment Scalar Value\\
& | & \text{i} \comment Index Value\\
& | & \shapevar \comment Shape Value \\
& | &  \text{x} \comment Variable Access \\
& | &  \stgvar{} \comment Reference Access\\
& | & \text{c} \comment Constants\\
& | & \lett{} x = \exprdps{} \inn{} \exprdps{} \comment (Non-Rec.) Let Binding\\
& | & \vifthenelse{\exprdps{}}{\exprdps{}}{\exprdps{}} \comment Conditional\\
& | & \withstg{\exprdps}{\stgvar}{\exprdps} \comment Memory Allocation\\
% & | & \varrs{\stgvar}{\vabs{\stgvar}{\exprdps}, ..., \vabs{\stgvar}{\exprdps}} \comment Vector Constructor\\
% \valvar & ::= & \vabs{\stgvar{} \vmore{\text{x}}}{\expr} \comment Storage-Aware Abstraction\\
% & | & \vabscard{\vmore{\text{x}}}{\expr} \comment Cardinality Abstraction \\
% & | & \stgval \comment Memory Location\\
% & | & \stgempty \comment Empty Memory Location\\
% & | & \text{n} \comment Scalar Value\\
% & | & \text{i} \comment Index Value\\
% & | & \shapevar \comment Shape Value\\
% & | & \varrsat{\stgvar}{...\varr{\text{n}, ..., \text{n}}...} \comment Vector Value\\
% \shapevar & ::= & \valcard \comment Cardinality Value \\
% & | & \cardvector{\shapevar}{\valcard} \comment Vector Shape Value \\
% & | & \cardempty \comment No Cardinality\\
\shapevar & ::= & \cardempty \comment Zero Cardinality\\
& | & \valcard \comment Cardinality Value \\
& | & \cardvector{\valcard}{\shapevar} \comment Vector Shape Value \\
\text{c} & ::= & \textit{[See Figure~\ref{fig:salaf_type_system}]} \\
\typetdps{} & ::= & \typemat{} | 
\typefun{\typetdps}{\typemat} |
\typebool \\
& | & \carddom \comment Shape Type\\
& | & \typestg \tab \tab \tab \tab \tab \comment Machine Address\\
\typemat{} & ::= & \typenum{} |
\typearray{\typemat{}}\\
\typenum & ::= & \typedouble{} |
\typeindex \\
\carddom{} & ::= & \typecard{} \comment Cardinality Type\\
& | & \typenestedcard{\typecard{}}{\carddom{}} \comment Vector Shape Type\\
\end{tabular}
\caption{The core \salafsharp{} syntax.}
\label{fig:salaf_core_syntax}
\end{figure}
\begin{figure*}[h]
\centering
\def\comment{\hfill -- }
% \begin{flushleft}
% \textbf{Type Syntax:}
% \end{flushleft}
% \begin{subfigure}[b]{.50\textwidth}
% \begin{tabular}{r c l}
% \typetdps{} & ::= & \typemat{} \comment Matrix Type\\
% & | & \typefun{\typetdps}{\typemat} \comment Function Types (No Currying)\\
% & | & \carddom \comment Shape Type\\
% & | & \typebool \comment Boolean Type\\
% & | & \typestg \tab \tab \tab \tab \tab \comment Machine Address\\
% \end{tabular}
% \end{subfigure}
% \begin{subfigure}[b]{.48\textwidth}
% \begin{tabular}{r c l}
% \typemat{} & ::= & \typenum{} \comment Numeric Type\\
% & | & \typearray{\typemat{}} \comment Vector, Matrix, ... Type\\
% \typenum & ::= & \typedouble{} \comment Scalar Type\\
% & | & \typeindex \comment Index Type\\
% \carddom{} & ::= & \typecard{} \comment Cardinality Type\\
% & | & \typenestedcard{\typecard{}}{\carddom{}} \comment Vector Shape Type\\

% \end{tabular}
% \end{subfigure}
% \\
\begin{flushleft}
\textbf{Typing Rules:}
\end{flushleft}
% \begin{subfigure}[b]{.48\textwidth}
% (T-App) $\infer{\Gamma \vdash \exprdpsind{0}: \typestg, \typefun{\typetdps}{\typemat} \tab \stgvar: \typestg \in \Gamma \tab \Gamma \vdash \vmore{\exprdps} : \vmore{\typetdps}}{\Gamma \vdash \exprdpsind{0}\ \stgvar{}\ \vmore{\exprdps}: \typemat}$ \\
% (T-Abs) $\infer{ \Gamma, \stgvar: \typestg  \cup  \vmore{\text{x}}: \vmore{\typetdps} \vdash \exprdps{}: \typemat }{\Gamma \vdash \lambda \stgvar \vmore{\text{x}}. \exprdps{}: \typestg, \typefun{\typetdps}{\typemat}}$ \\
% \end{subfigure}
\begin{subfigure}[b]{.48\textwidth}
(T-Alloc) $\infer{\Gamma \vdash \exprdpsind{0}: \typecard \tab \tab \Gamma, \stgvar: \typestg \vdash  \exprdpsind{1}: \typemat}{\text{\withstg{\exprdpsind{0}}{\stgvar}{\exprdpsind{1}}: \typemat}}$\\
% (T-Vectors) $\infer{\Gamma, \stgvarind{i}: \typestg \vdash \exprdpsind{i}: \typemat \tab \stgvar{}: \typestg \in \Gamma}{\Gamma \vdash \varrs{\stgvar}{\vabs{\stgvarind{1}}{\exprdpsind{1}}, ..., \vabs{\stgvarind{\valcard}}{\exprdpsind{\valcard}}}: \typearray{\typemat}}$\\
\end{subfigure}
% (T-Builds) $\infer{\Gamma \vdash \exprind{0}: \typecard \tab \Gamma \vdash \exprind{1}: \typefunone{\typestg, \typeindex}{\typemat} \tab \stgvar: \typestg \in \Gamma}{\Gamma \v\dash \text{\vbuilds{\stgvar}{\exprind{0}}{\exprind{1}}: \typearray{\typemat}}}$\\
% (T-Folds) $\infer{\Gamma \vdash \exprind{1}: \typearray{$\typemat_1$} \tab \Gamma \vdash \exprind{2}: \text{$\typemat_2$}  \tab \Gamma \vdash \exprind{3}: \typefunone{\typestg, \typemat_2, \typemat_1}{\typemat_2} \tab \stgvar{}: \typestg \in \Gamma}{\text{\vfolds{\stgvar}{\exprind{2}}{\exprind{3}}{\exprind{1}}: $\typemat_2$}}$\\
\vspace{.5cm}
\begin{flushleft}
\begin{subfigure}[t]{.50\textwidth}
\textbf{Vector Function Constants:}
\\
\setlength\tabcolsep{1.5pt}
\begin{tabular}{l c l}
\stgvarpostfix{\vcbuild} &:& \typestg{}, \typecard{}, 
(\typefunone{\typestg, \typeindex}{\typemat}),
\\
& &
\multicolumn{1}{l}{
\tab \tab \tab
\typecard{},
(\typefunone{\typecard}{\typeshape})
}
\\
& &
\multicolumn{1}{c}{
\typefunone{}{\typearray{\typemat}}
}
\\
\stgvarpostfix{\viteratek} &:&
\typestg{},
(\typefunone{\typestg, $\typemat$, \typeindex}{$\typemat$}),
$\typemat$,
\typecard{},
\\
& &
\multicolumn{1}{l}{
\tab \tab \tab
(\typefunone{\typeshape, \typecard}{\typeshape}),
\typeshape{},
\typecard{}
}
\\
& &
\multicolumn{1}{c}{
\typefunone{}{\typemat}
}
\\
\stgvarpostfix{\vcget} &:&
\typestg{},
\typearray{\typemat},
\typeindex{},
\\
& &
\multicolumn{1}{l}{
\tab \tab \tab
\typefunone{\typeshape{}, \typecard{}}{\typemat}
}
\\
\stgvarpostfix{\vclength} &:& 
\typefunone{\typestg{},
\typearray{\typemat},
\typeshape
}{\typecard}
\\
% \stgvarpostfix{\vcfold} &:&
% \typefunone{\typestg}{
% \typefunone{(\typefunone{\typestg, $\typemat_2$, $\typemat_1$}{$\typemat_2$})}{}
% }\\
% & &
% \multicolumn{1}{c}{
% \typefunone{\typefunone{$\typemat_2$}{\typearray{$\typemat_1$}}}{$\typemat_2$}
% }
% \\
\vcopys{}{} &:& \typefunone{\typestg{},
\typearray{\typemat}}{
  \typearray{\typemat}
}
\end{tabular}
\end{subfigure}
\begin{subfigure}[t]{.48\textwidth}
\textbf{Scalar Function Constants:}
\\
\begin{tabular}{l c l}
\multicolumn{3}{l}{\textit{DPS version of \lafsharp{} Scalar Constants (See  Figure~\ref{fig:laf_type_system}).}} \\
\stgoffset{} &:& \typefunone{\typestg{}, \carddom}{\typestg} \\
\vcvecshape{} & : & \typefunone{\typecard{}, \carddom{}}{\typenestedcard{\typecard}{\carddom}} \\
\cardvectorsize{} &:& \typefunone{\typenestedcard{\typecard}{\carddom}}{\typecard} \\
\cardvectorelem{} &:& \typefunone{\typenestedcard{\typecard}{\carddom}}{\carddom} \\
\cardwidth{} &:& \typefunone{\carddom}{\typecard} \\
\end{tabular}\\
\textbf{Syntactic Sugars:} \\
\vgets{\stgvar}{\exprdpsind{0}}{\exprdpsind{1}}
=
\stgvarpostfix{\vcget{}} \stgvar{} \exprdpsind{0} \exprdpsind{1}
\tab
\vlength{\exprdps}
=
\stgvarpostfix{\vclength{}} \stgempty{} \exprdps{}\\
\cardvector{\exprdpsind{0}}{\exprdpsind{1}}
=
\vcvecshape{} \exprdpsind{0} \exprdpsind{1} 
\\
\textit{for all binary ops $bop$:}
\exprind{1} $bop$ \exprind{2} = $bop$ \stgempty{} \exprind{1} \exprind{2}
\end{subfigure}
\\
% \vspace{.5cm}
% \textbf{Syntactic Sugars:} \\
% \begin{subfigure}[b]{.50\textwidth}
% \begin{tabular}{l c l}
% \vgets{\stgvar}{\exprdpsind{0}}{\exprdpsind{1}}
% &=&
% \stgvarpostfix{\vcget{}} \stgvar{} \exprdpsind{0} \exprdpsind{1}\\
% \vlength{\exprdps}
% &=&
% \stgvarpostfix{\vclength{}} \stgempty{} \exprdps{}\\
% \end{tabular}
% \end{subfigure}
% \begin{subfigure}[b]{.48\textwidth}
% \begin{tabular}{l c l}
% % \vfolds{\stgvar}{\exprind{2}}{\exprind{1}}{\exprind{3}}
% % &=& 
% % \stgvarpostfix{\vcfold{}} \stgvar{} \exprind{1} \exprind{2} \exprind{3}\\
% \cardvector{\exprdpsind{0}}{\exprdpsind{1}}
% &=&
% \vcvecshape{} \exprdpsind{0} \exprdpsind{1} \\
% $\forall bop.$ \exprind{1} $bop$ \exprind{2} &=& $bop$ \stgempty{} \exprind{1} \exprind{2}
% \end{tabular}
% \end{subfigure}
\end{flushleft}
\caption{The type system and built-in constants of \salafsharp{}}
\label{fig:salaf_type_system}
\end{figure*}


Thus motivated, we define a new intermediate language, \salafsharp{}, in which memory
allocation and deallocation is explicit.  \salafsharp{} uses
\emph{destination-passing style}: every array-returning function receives as its first parameter a pointer to memory in which to store the result array.
No function  allocates the storage needed for its result; instead the responsibility of allocating
and deallocating the output storage of a function is  given to the caller of that function. Similarly, all the storage allocated inside a function can be deallocated as soon as the function returns its result. 

Destination passing style is a standard programming idiom in C.  For example, the C standard library procedures that return a string (e.g. \code{strcpy}) expect the caller to provide storage for the result. This gives the programmer full control over memory management for string values. Other languages have exploited destination-passing style during compilation~\cite{Henriksen:2014:SSH:2636228.2636238, Henriksen:2014:BCI:2627373.2627388}.


% Every storage is allocated before being passed to a function. This way, that function does need to worry about freeing that storage space, and should only manage the storage allocated inside its body.

\subsection{The \salafsharp{} language}

The syntax of \salafsharp{} is shown in Figure~\ref{fig:salaf_core_syntax}, while its type system is in Figure~\ref{fig:salaf_type_system}. 
The main additional construct in this language is the one for allocating a particular amount of storage space \vallocterm{} t1 (\vabs{\stgvarind{}}{t2}). In this construct
t1 is an expression that evaluates to the size (in bytes) that is required for storing
the result of evaluating t2. This storage is available in the lexical scope of the 
lambda parameter, \emph{and is deallocated outside this scope}. 
The previous example can be written in the following way in \salafsharp{}:

\begin{figure}[H]
\hfill\begin{minipage}{.75\textwidth}\raggedright
f = \vabs{\stgvarind{1} vec1 vec2}{} \vallocterm{} (vecBytes vec1) (\vabs{\stgvarind{2}}{}
\\
\tabt \tabt
vectorNorm\_dps \stgempty{} (vectorAdd\_dps \stgvarind{2} vec1 vec2)
\\
\tabt )
\end{minipage}\hfill
\end{figure}
Each lambda abstraction typically takes an additional parameter which specifies the storage space that 
is used for its result.  Furthermore, every application should be applied to an additional parameter 
which specifies the memory location of the return value in the case of an array-returning function. 
However, a scalar-returning function is applied to a dummy empty memory location, specified by \stgempty{}. 
In this example, the number of bytes allocated for the memory location \stgvarind{2} is specified
by the expression (vecBytes vec1) which computes the number of bytes of the array vec1.

%Every expression which results in the creation of a vector requires passing the output storage space. This storage space is created using the \cod{alloc} construct in the call site of the functions. This is one of the main differences of DPS with the traditional region-based memory management.

