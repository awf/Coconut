\begin{figure*}[h]
\centering
\def\comment{\hfill -- }
% \begin{flushleft}
% \textbf{Type Syntax:}
% \end{flushleft}
% \begin{subfigure}[b]{.50\textwidth}
% \begin{tabular}{r c l}
% \typetdps{} & ::= & \typemat{} \comment Matrix Type\\
% & | & \typefun{\typetdps}{\typemat} \comment Function Types (No Currying)\\
% & | & \carddom \comment Shape Type\\
% & | & \typebool \comment Boolean Type\\
% & | & \typestg \tab \tab \tab \tab \tab \comment Machine Address\\
% \end{tabular}
% \end{subfigure}
% \begin{subfigure}[b]{.48\textwidth}
% \begin{tabular}{r c l}
% \typemat{} & ::= & \typenum{} \comment Numeric Type\\
% & | & \typearray{\typemat{}} \comment Vector, Matrix, ... Type\\
% \typenum & ::= & \typedouble{} \comment Scalar Type\\
% & | & \typeindex \comment Index Type\\
% \carddom{} & ::= & \typecard{} \comment Cardinality Type\\
% & | & \typenestedcard{\typecard{}}{\carddom{}} \comment Vector Shape Type\\

% \end{tabular}
% \end{subfigure}
% \\
\begin{flushleft}
\textbf{Typing Rules:}
\end{flushleft}
% \begin{subfigure}[b]{.48\textwidth}
% (T-App) $\infer{\Gamma \vdash \exprdpsind{0}: \typestg, \typefun{\typetdps}{\typemat} \tab \stgvar: \typestg \in \Gamma \tab \Gamma \vdash \vmore{\exprdps} : \vmore{\typetdps}}{\Gamma \vdash \exprdpsind{0}\ \stgvar{}\ \vmore{\exprdps}: \typemat}$ \\
% (T-Abs) $\infer{ \Gamma, \stgvar: \typestg  \cup  \vmore{\text{x}}: \vmore{\typetdps} \vdash \exprdps{}: \typemat }{\Gamma \vdash \lambda \stgvar \vmore{\text{x}}. \exprdps{}: \typestg, \typefun{\typetdps}{\typemat}}$ \\
% \end{subfigure}
\begin{subfigure}[b]{.48\textwidth}
(T-Alloc) $\infer{\Gamma \vdash \exprdpsind{0}: \typecard \tab \tab \Gamma, \stgvar: \typestg \vdash  \exprdpsind{1}: \typemat}{\text{\withstg{\exprdpsind{0}}{\stgvar}{\exprdpsind{1}}: \typemat}}$\\
% (T-Vectors) $\infer{\Gamma, \stgvarind{i}: \typestg \vdash \exprdpsind{i}: \typemat \tab \stgvar{}: \typestg \in \Gamma}{\Gamma \vdash \varrs{\stgvar}{\vabs{\stgvarind{1}}{\exprdpsind{1}}, ..., \vabs{\stgvarind{\valcard}}{\exprdpsind{\valcard}}}: \typearray{\typemat}}$\\
\end{subfigure}
% (T-Builds) $\infer{\Gamma \vdash \exprind{0}: \typecard \tab \Gamma \vdash \exprind{1}: \typefunone{\typestg, \typeindex}{\typemat} \tab \stgvar: \typestg \in \Gamma}{\Gamma \v\dash \text{\vbuilds{\stgvar}{\exprind{0}}{\exprind{1}}: \typearray{\typemat}}}$\\
% (T-Folds) $\infer{\Gamma \vdash \exprind{1}: \typearray{$\typemat_1$} \tab \Gamma \vdash \exprind{2}: \text{$\typemat_2$}  \tab \Gamma \vdash \exprind{3}: \typefunone{\typestg, \typemat_2, \typemat_1}{\typemat_2} \tab \stgvar{}: \typestg \in \Gamma}{\text{\vfolds{\stgvar}{\exprind{2}}{\exprind{3}}{\exprind{1}}: $\typemat_2$}}$\\
\vspace{.5cm}
\begin{flushleft}
\begin{subfigure}[t]{.50\textwidth}
\textbf{Vector Function Constants:}
\\
\setlength\tabcolsep{1.5pt}
\begin{tabular}{l c l}
\stgvarpostfix{\vcbuild} &:& \typestg{}, \typecard{}, 
(\typefunone{\typestg, \typeindex}{\typemat}),
\\
& &
\multicolumn{1}{l}{
\tab \tab \tab
\typecard{},
(\typefunone{\typecard}{\typeshape})
}
\\
& &
\multicolumn{1}{c}{
\typefunone{}{\typearray{\typemat}}
}
\\
\stgvarpostfix{\viteratek} &:&
\typestg{},
(\typefunone{\typestg, $\typemat$, \typeindex}{$\typemat$}),
$\typemat$,
\typecard{},
\\
& &
\multicolumn{1}{l}{
\tab \tab \tab
(\typefunone{\typeshape, \typecard}{\typeshape}),
\typeshape{},
\typecard{}
}
\\
& &
\multicolumn{1}{c}{
\typefunone{}{\typemat}
}
\\
\stgvarpostfix{\vcget} &:&
\typestg{},
\typearray{\typemat},
\typeindex{},
\\
& &
\multicolumn{1}{l}{
\tab \tab \tab
\typefunone{\typeshape{}, \typecard{}}{\typemat}
}
\\
\stgvarpostfix{\vclength} &:& 
\typefunone{\typestg{},
\typearray{\typemat},
\typeshape
}{\typecard}
\\
% \stgvarpostfix{\vcfold} &:&
% \typefunone{\typestg}{
% \typefunone{(\typefunone{\typestg, $\typemat_2$, $\typemat_1$}{$\typemat_2$})}{}
% }\\
% & &
% \multicolumn{1}{c}{
% \typefunone{\typefunone{$\typemat_2$}{\typearray{$\typemat_1$}}}{$\typemat_2$}
% }
% \\
\vcopys{}{} &:& \typefunone{\typestg{},
\typearray{\typemat}}{
  \typearray{\typemat}
}
\end{tabular}
\end{subfigure}
\begin{subfigure}[t]{.48\textwidth}
\textbf{Scalar Function Constants:}
\\
\begin{tabular}{l c l}
\multicolumn{3}{l}{\textit{DPS version of \lafsharp{} Scalar Constants (See  Figure~\ref{fig:laf_type_system}).}} \\
\stgoffset{} &:& \typefunone{\typestg{}, \carddom}{\typestg} \\
\vcvecshape{} & : & \typefunone{\typecard{}, \carddom{}}{\typenestedcard{\typecard}{\carddom}} \\
\cardvectorsize{} &:& \typefunone{\typenestedcard{\typecard}{\carddom}}{\typecard} \\
\cardvectorelem{} &:& \typefunone{\typenestedcard{\typecard}{\carddom}}{\carddom} \\
\cardwidth{} &:& \typefunone{\carddom}{\typecard} \\
\end{tabular}\\
\textbf{Syntactic Sugars:} \\
\vgets{\stgvar}{\exprdpsind{0}}{\exprdpsind{1}}
=
\stgvarpostfix{\vcget{}} \stgvar{} \exprdpsind{0} \exprdpsind{1}
\tab
\vlength{\exprdps}
=
\stgvarpostfix{\vclength{}} \stgempty{} \exprdps{}\\
\cardvector{\exprdpsind{0}}{\exprdpsind{1}}
=
\vcvecshape{} \exprdpsind{0} \exprdpsind{1} 
\\
\textit{for all binary ops $bop$:}
\exprind{1} $bop$ \exprind{2} = $bop$ \stgempty{} \exprind{1} \exprind{2}
\end{subfigure}
\\
% \vspace{.5cm}
% \textbf{Syntactic Sugars:} \\
% \begin{subfigure}[b]{.50\textwidth}
% \begin{tabular}{l c l}
% \vgets{\stgvar}{\exprdpsind{0}}{\exprdpsind{1}}
% &=&
% \stgvarpostfix{\vcget{}} \stgvar{} \exprdpsind{0} \exprdpsind{1}\\
% \vlength{\exprdps}
% &=&
% \stgvarpostfix{\vclength{}} \stgempty{} \exprdps{}\\
% \end{tabular}
% \end{subfigure}
% \begin{subfigure}[b]{.48\textwidth}
% \begin{tabular}{l c l}
% % \vfolds{\stgvar}{\exprind{2}}{\exprind{1}}{\exprind{3}}
% % &=& 
% % \stgvarpostfix{\vcfold{}} \stgvar{} \exprind{1} \exprind{2} \exprind{3}\\
% \cardvector{\exprdpsind{0}}{\exprdpsind{1}}
% &=&
% \vcvecshape{} \exprdpsind{0} \exprdpsind{1} \\
% $\forall bop.$ \exprind{1} $bop$ \exprind{2} &=& $bop$ \stgempty{} \exprind{1} \exprind{2}
% \end{tabular}
% \end{subfigure}
\end{flushleft}
\caption{The type system and built-in constants of \salafsharp{}}
\label{fig:salaf_type_system}
\end{figure*}
