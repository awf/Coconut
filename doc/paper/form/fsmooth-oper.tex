
\begin{figure*}
\centering
\begin{subfigure}[t]{\columnwidth}

\def\comment{\hfill -- }
\begin{flushleft}
\textbf{Value Syntax:}\\
\end{flushleft}
\centering
\begin{tabular}{r c l}
\valvar & ::= & $\lambda \vmore{\text{x}}. \expr$ \comment Abstraction\\
& | & \text{n} \comment Scalar Value\\
& | & \text{i} \comment Index Value\\
& | & \valcard \comment Cardinality Value\\
& | & \varr{\valvar, ..., \valvar} \comment Vector Value\\
\end{tabular}
\\
\begin{flushleft}
\textbf{Computation Rules:}\\
\end{flushleft}
\begin{tabular}{r c l}
% (E-IfTrue) &
\vifthenelse{true}{\exprind{2}}{\exprind{3}} 
&\evalsto&
\exprind{2} \\
% (E-IfFalse) &
\vifthenelse{false}{\exprind{2}}{\exprind{3}} 
&\evalsto&
\exprind{3} \\
% (E-AppBeta) &
$\lambda$ \vmore{\text{x}}. \exprind{1}\ \vmore{\text{v}} 
&\evalsto&
\exprind{1}[\vmore{\text{x}} $\mapsto$ \vmore{\text{v}}] \\
% (E-LetBeta)& 
\text{\lett{} x = \text{v} \inn{} \exprind{2}} 
&\evalsto&
\exprind{2}[\text{x} $\mapsto$ \text{v}]\\
% (E-BuildEval)&
\vbuild{\valcard}{\text{v}} 
&\evalsto&
\varr{\text{v}\ 1, ..., \text{v}\ \valcard} \\
% (E-GetEval)&
\vget{\varr{\valvarind{1}, ..., \valvarind{\valcard}}}{\text{i}} 
&\evalsto&
\valvarind{\text{i}} \\
% (E-LengthEval)&
\vlength{\varr{\valvarind{1}, ..., \valvarind{\valcard}}}
&\evalsto&
\valcard \\
% (E-FoldStep)&
$\forall \valcard > 0.$ \viterate{\valvarind{1}}{\valvarind{2}}{\valcard} 
&\evalsto& \viterate{\valvarind{1}}{(\valvarind{1} \valvarind{2} (\valcard{} - 1))}{(\valcard{} - 1)}
\\
% (E-FoldLast)&
\viterate{\valvarind{1}}{\valvarind{2}}{0} 
&\evalsto&
\valvarind{2}\\
\end{tabular}
% \subcaption{Computation Rules}

\end{subfigure}
\begin{subfigure}[t]{\columnwidth}
% \newcommand{\comment}[1]{\hfill -- #1}
\newcommand{\comment}[1]{}
\begin{flushleft}
\textbf{Evaluation Context:}\\
\end{flushleft}
\centering
\begin{tabular}{r c l}
\evalctx{} & ::= & \comment{Evaluation Context}\\
& | & [] \comment{ Evaluation Hole}\\
& | & \vifthenelse{\evalctx}{\exprind{2}}{\exprind{3}} \comment{If Condition} \\
& | & \vapp{\evalctx}{$\vmore{\expr}$} \comment{Application Function}\\
& | & \vapp{\valvarind{0}}{\valvarind{1} ... \valvarind{k-1} \evalctx{} \exprind{k+1} ... \exprind{N}} \comment{Application Arguments}\\
& | & \vlet{\text{x}}{\evalctx}{\exprind{1}} \comment{Let Binding}\\
& | & \varr{\valvarind{1}, ..., \valvarind{k-1}, \evalctx{}, \exprind{k+1}, ..., \exprind{N}} \comment{Vector Construct}\\
% & | & \vbuild{\evalctx}{\exprind{1}} \comment{Build Size}\\
% & | & \vbuild{\valvarind{0}}{\evalctx} \comment{Build Function}\\
% & | & \vget{\evalctx}{\exprind{1}} \comment{Get Vector}\\
% & | & \vget{\valvarind{0}}{\evalctx} \comment{Get Index}\\
% & | & \vlength{\evalctx} \comment{Length}\\
% & | & \vfold{\exprind{2}}{\exprind{3}}{\evalctx} \comment{Fold Range}\\
% & | & \vfold{\evalctx}{\exprind{3}}{\valvarind{1}} \comment{Fold Init}\\
% & | & \vfold{\valvarind{2}}{\evalctx}{\valvarind{1}} \comment{Fold Function}
\\ \\
& & %\hspace*{0.2cm}
\infer{\expr{} \evalsto \expr{}\vprime}{\evalctx[\expr{}] \evalsto \evalctx[\expr{}\vprime]
}
\end{tabular}

% \subcaption{Congruence Rules}
\end{subfigure}

\caption{Operational Semantics of \lafsharp{}}
\label{fig:laf_red}
\end{figure*}