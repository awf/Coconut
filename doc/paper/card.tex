\subsection{Shape translation}
\label{sec:card}\label{sec:shapetrans}

\begin{figure*}
\centering
% \def\comment{\hfill -- }
% \begin{tabular}{r c l}
% \carddom{} & ::= & \carddommat{} \comment Matrix Cardinality \\
% & | & \typefun{\carddom}{\carddommat} \comment Function Cardinality \\
% & | & \cardempty \comment No Cardinality\\
% & | & \typecard \comment Cardinality\\
% \\
% \carddommat{} & ::= & \carddouble{} \comment Scalar Width\\
% & | & \cardvector{\carddommat{}}{\typecard} \comment Vector, Matrix, ... Cardinality\\
% \end{tabular}
% % \infer{\expr{}=\exprind{0}\ \vmore{\expr}\tab \expr{}: \typemat \tab \typemat \neq \typedouble}
% \vspace{1cm}

\begin{tabular}{l r c l}
% & \cardtrans{\lafsharp{} Expression} &=&
% \shapefsmooth{} Expression
& \cardtrans{\expr} &=&
\exprshape{}
\\ \\
(S-App) & 
\cardtrans{\vapp{\exprind{0}}{\exprind{1} ... \exprind{k}}
}&=&
\vapp{\cardtrans{\exprind{0}}}{\cardtrans{\exprind{1}} ... \cardtrans{\exprind{k}}}\\
(S-Abs) & 
\cardtrans{\vabs{$x_1$: $T_1$, ..., $x_k$: $T_k$}{\expr}
}&=&
\vabscard{\cardvarpostfix{$x_1$}: \cardtranstype{$T_1$}, ..., \cardvarpostfix{$x_k$}: \cardtranstype{$T_k$}}{\cardtrans{\expr}}\\
(S-Var) & 
\cardtrans{\text{x}}&=&
\cardvarpostfix{\text{x}}\\
(S-Let) & 
\cardtrans{\vlet{\text{x}}{\exprind{1}}{\exprind{2}}}&=&
\vlet{\cardvarpostfix{\text{x}}}{\cardtrans{\exprind{1}}}{\cardtrans{\exprind{2}}}\\
(S-If) & 
\cardtrans{\vifthenelse{\exprind{1}}{\exprind{2}}{\exprind{3}}}&=&
$\begin{cases}
    \cardtrans{\exprind{2}}       & \qquad \qquad \cardtrans{\exprind{2}} \shpeq{} \cardtrans{\exprind{3}}\\
    \text{Compilation Error!}  & \qquad \qquad \cardtrans{\exprind{2}} \shpneq{} \cardtrans{\exprind{3}}\\
  \end{cases}
$
\\
% (C-Vector) & \cardtrans{\varr{\exprind{1}, ..., \exprind{\valcard}}} &=&
% \cardvector{\cardtrans{\exprind{1}}}{\valcard} \\
% (S-Vector) &  \cardtrans{\varr{\exprind{1}, ..., \exprind{\valcard}}} &=&
% $\begin{cases}
%     \valcard{}       & \quad \forall k.\ \exprind{k}: \typenum{}\\
%       & \quad \forall k.\ \exprind{k}: \typearray{\typemat{}}
%     \\
%     \cardvector{\cardtrans{\exprind{1}}}{\valcard} & \quad \text{and}
%     \\
%     & \quad \forall k,k'.\ \cardtrans{\exprind{k}}\shpeq{}\cardtrans{\exprind{k'}}\\
%     \text{Compilation Error!} & \quad \text{otherwise}
%   \end{cases}
% $
% (S-Vector) &  \cardtrans{\varr{\exprind{1}, ..., \exprind{\valcard}}} &=&
% $\begin{cases}
%     \cardvector{\valcard}{\cardtrans{\exprind{1}}} & \quad \forall k,k'.\ \cardtrans{\exprind{k}}\shpeq{}\cardtrans{\exprind{k'}}
%     \\
%     \text{Compilation Error!} & \qquad \qquad \text{otherwise}
%   \end{cases}
% $
%  \\
% (C-Matrix) & \exprind{k}: \typearray{\typemat{}} $\vdash$ \cardtrans{\varr{\exprind{1}, ..., \exprind{\valcard}}} &=&
% \cardvector{\cardtrans{\exprind{1}}}{\valcard} \\
% (C-Build) &
% \cardtrans{\vbuild{\exprind{0}}{\exprind{1}}} &=&
% \cardvector{(\vapp{\cardtrans{\exprind{1}}}{\cardempty})}{\cardtrans{\exprind{0}}} \\
% (C-Get) &
% \cardtrans{\vget{\exprind{0}}{\exprind{1}}} &=&
% \cardvectorelem{\cardtrans{\exprind{0}}} \\
% (C-Length) &
% \cardtrans{\vlength{\exprind{0}}} &=&
% \cardvectorsize{\cardtrans{\exprind{0}}} \\
% (C-Fold) &
% \cardtrans{\vfold{\exprind{2}}{\exprind{3}}{\exprind{1}}} &=&
% \cardtrans{\exprind{2}} \\
\\
(S-ExpNum) &
\expr{}: \typenum{} $\vdash$
\cardtrans{\expr{}} &=&
\cardempty \\
(S-ExpBool) &
\expr{}: \typebool{} $\vdash$
\cardtrans{\expr{}} &=&
\cardempty \\
(S-ValCard) &
\cardtrans{\text{N}} &=&
\text{N} \\
(S-AddCard) &
\cardtrans{\exprind{0} \vcaddcard{} \exprind{1}} &=&
\cardtrans{\exprind{0}} \vcaddcard{} \cardtrans{\exprind{1}} \\
(S-MulCard) &
\cardtrans{\exprind{0} \vcmulcard{} \exprind{1}} &=&
\cardtrans{\exprind{0}} \vcmulcard{} \cardtrans{\exprind{1}} \\
(S-Build) &
\cardtrans{\vbuild{\exprind{0}}{\exprind{1}}}
 &=&
\cardvector{\cardtrans{\exprind{0}}}{(\vapp{\cardtrans{\exprind{1}}}{\cardempty})} \\
(S-Get) &
\cardtrans{\vget{\exprind{0}}{\exprind{1}}}
&=&
\cardvectorelem{\cardtrans{\exprind{0}}} \\
(S-Length) &
\cardtrans{\vlength{\exprind{0}}}
&=&
\cardvectorsize{\cardtrans{\exprind{0}}} \\
(S-Reduce) &
\cardtrans{
\viteratek{} \exprind{1} \exprind{2} \exprind{3}
} &=&
$\begin{cases}
    \cardtrans{\exprind{2}}       & \quad \forall n. \cardtrans{\exprind{1} \exprind{2} n} \shpeq{} \cardtrans{\exprind{2}}\\
    \text{Compilation Error!}  & \qquad \qquad \text{otherwise}\\
  \end{cases}
$
\\ \\
& \cardtranstype{\typet{}} &=& \typetshape{}
\\ \\

(ST-Fun) &
\cardtranstype{\typefunone{\typeind{1}, \typeind{2}, ..., \typeind{k}}{\typemat}} &=&
\typefunone{\cardtranstype{\typeind{1}}, \cardtranstype{\typeind{2}}, ..., \cardtranstype{\typeind{k}}}{\cardtranstype{\typemat}} \\
(ST-Num) &
\cardtranstype{\typenum} &=&
\typecard \\
(ST-Bool) &
\cardtranstype{\typebool} &=&
\typecard \\
(ST-Card) &
\cardtranstype{\typecard} &=&
\typecard \\
% (CT-Vector) &
% \cardtranstype{\typearray{\typemat}} &=&
% \cardvector{\cardtranstype{\typemat}}{\typecard} \\
% (ST-Vector) &
% \cardtranstype{\typearray{\typenum}} &=&
% {\typecard} \\
(ST-Vector) &
\cardtranstype{\typearray{\typemat}} &=&
\cardvector{\typecard}{\cardtranstype{\typemat}} \\
% \\
% (W-NoCard) &
% \cardwidth{\cardempty} &=&
% 0
% \\
% (W-Card) &
% \cardwidth{\valcard} &=&
% \valcard
% \\
% (W-Scalar) &
% \cardwidth{\carddouble} &=&
% 4
% \\
% (W-Vector) &
% \cardwidth{\cardvector{c}{\valcard}} &=&
% \cardwidth{c} $\times$ \valcard
% \\
\end{tabular}
\caption{Shape Translation of \lafsharp{}}
\label{fig:laf_card}
\end{figure*}

As we have seen, rule (D-Let) relies on the \emph{shape translation} of the right
hand side.  This translation is given in Figure~\ref{fig:laf_card}.
If $\expr$ has type $\typet$, then $\cardtrans{\expr}$ is an expression 
of type $\cardtranstype{\typet}$ that gives the shape of $\expr$.
This expression can always be evaluated without allocation.

A \emph{shape} is an expression of type $\carddom{}$ (Figure~\ref{fig:salaf_core_syntax}),
whose values are given by $\shapevar$ in that Figure.  There are three cases to consider

First, a scalar value has shape $\cardempty$ (rules (S-ExpNum), (S-ExpBool)).

Second, when $\expr$ is an array, $\cardtrans{\expr}$ gives the shape of the array as
a nested tuple, such as $(3,(4,\cardempty))$ for a 3-vector of 4-vectors.
So the ``shape'' of an array specifies the cardinality of each dimension.

Finally, when $\expr$ is a function, $\cardtrans{\expr}$ is a function that takes the shapes of its arguments and returns the shape of its result. 
You can see this directly in rule (S-App): to compute the shape of (the result of) a call, apply the shape-translation of the function to the shapes of the arguments.
This is possible because \lafsharp{} programs do not allow the programmer
to write a function whose result size depends on the contents of its input array.

What is the shape-translation of a function $\text{f}$?  Remembering that every in-scope variable $\text{f}$ has become a pair of variables one for the value and one for the shape, we can simply use the latter, $\cardvarpostfix{\text{f}}$, as we see in rule (S-Var).

For arrays, could the shape be simply the number of bytes required for the array,
rather than a nested tuple?  No.
Consider the following function, which returns the first row of its argument matrix:

firstRow = \vabs{m: \typearray{\typearray{\typedouble{}}}}{\vget{m}{0}}

The shape translation of firstRow, namely \cardvarpostfix{firstRow}, is given the shape of m, and must produce the
shape of m's first row.  It cannot do that given only the number of bytes in m; it must know how many rows
and columns it has.  But by defining shapes as a nested tuple, it becomes easy: see rule (S-Get).

The shape of the result of the iteration construct (\viteratek) requires the shape of 
the state expression to remain the same across iterations. Otherwise the compiler
produce an error, as it is shown in rule (S-Reduce).  

The other rules are straightforward.  \emph{The key point is this: by translating every in-scope variable, including functions, into a pair of variables, we can give a \emph{compositional} account of shape translation, even in a higher order language.}




